\documentclass[pdf,8pt]{beamer}

\usepackage[utf8]{inputenc}
\usepackage{graphicx}       % Images
\usepackage{xcolor}         % Change Colors
%\usepackage[font={scriptsize}]{caption}   % Adjust caption fonts
\usepackage{caption}
\captionsetup[figure]{font=small}
%\usetheme{Madrid}
\usepackage{multimedia}     % Movies!

% Title
\title{Visualizing High Performance Computing Data\\
    \large{Earthquake Simulation of Hayward Fault: SW4}\\
        \tiny{(In 3 minutes or less)}}
\author{Steven Walton\\
        \small University of Oregon}
\date{CIS 541: Computer Graphics Final \\18 March 2019}

% Background
\usebackgroundtemplate
{
    \includegraphics[width=\paperwidth,height=\paperheight]{UOTemplate.png}
}

\definecolor{UOYellow}{HTML}{FDCB00}

% Formatting
\setbeamercolor{title}{fg=white}
\setbeamercolor{normal text}{fg=white}
%\setbeamercolor{frametitle}{fg=white}
\setbeamercolor{structure}{fg=white}        % Adjusts figures and frame titles
%\setbeamercolor{palette primary}{fg=white}
%\setbeamercolor{palette secondary}{fg=white}
%\setbeamercolor{section in toc}{fg=white}
%\setbeamercolor{subsection in head/foot}{fg=white}


%\newcommand{\mytitle}[1]{\setcolor{bg=UOYellow,fg=green}\frametitle{{#1}}}
\newcommand\Quadrents[4]{
    \begin{minipage}[b][0.35\textheight][t]{0.49\textwidth}#1\end{minipage}\hfill
    \begin{minipage}[b][0.35\textheight][t]{0.49\textwidth}#2\end{minipage}\\[0.5em]
    \begin{minipage}[b][0.35\textheight][t]{0.49\textwidth}#3\end{minipage}\hfill
    \begin{minipage}[b][0.35\textheight][t]{0.49\textwidth}#4\end{minipage}
}

\newcommand\Split[2]{
    \begin{minipage}[b][0.35\textheight][t]{0.8\textwidth}#1\end{minipage}\\[2.0em]
    \begin{minipage}[b][0.35\textheight][t]{0.8\textwidth}#2\end{minipage}
}

\begin{document}

% 1
\frame{\titlepage}

% 2
\begin{frame}
\frametitle{What are we modeling?}

\begin{columns}
    % Left
    \begin{column}{0.49\paperwidth}
    \begin{itemize}
    \item Lawrence Livermore (LLNL) and Lawrence Berkeley (LBNL) National Labs are working
          on earthquake modeling within the Bay area. 
    \item United States Geologic Survey (USGS) predicts a 1 in 3 chance of a rupture
          with $\geq$ 6.7 magnitude in the next 30 years.
    \item LLNL and LBNL have created new earthquake modeling sims that are 4-8x
          more resolved than previous models.
    \item Scientists typically look at only 2D slices of data
    \item<2> Read more at: https://www.llnl.gov/news/hayward-fault-earthquake-simulations-increase-fidelity-ground-motions
    \end{itemize}
    \end{column}

    % Right
    \begin{column}{0.49\paperwidth}
    \begin{figure}[ht]
        \vspace{-4em}
        \begin{center}
            \includegraphics[width=\textwidth]{hayward_fault.png}
            \caption{Hayward Fault Line (source: Berkeley Seismology Lab)}
        \end{center}
    \end{figure}
    \end{column}
\end{columns}
\end{frame}

% 3
\begin{frame}
\frametitle{In Situ}
\begin{columns}
    % Left
    \begin{column}{0.69\paperwidth}
    \begin{itemize}
        \item So much data is being generated at once that it is impossible to save
              all of it and do analysis Post Hoc (after the fact)
        \item We need to analyze data as it is being generated, we call this
              In Situ.
        \item Still need to compress data so we can visualize it fast enough.
        \item We can use Visit or Ascent
        \item Visit is great for post hoc methods
        \item Ascent is built for in situ methods and with a small modification of
              code we can enable in situ techniques.
    \end{itemize}
    \end{column}
% Right
    \begin{column}{0.3\paperwidth}
    %\vspace{-3em}
    \begin{figure}[ht]
        \includegraphics[width=\textwidth]{aftershock.png}
    \caption{Aftershocks of M=7.1 Earthquake (source: Temblor)}
    \end{figure}
    \end{column}
\end{columns}
\end{frame}

\begin{frame}
\begin{columns}
    \begin{column}{0.34\paperwidth}
        \begin{itemize}
            \item<1-> Let's open up Visit and plot a single time step
            \item<2-> Let's try to apply some operators
            \item<4-> ... Memory is full
            \item<5-> ... Computer crashed
            \item<6-> ...
            \item<7-> ... Let's go to Alaska
        \end{itemize}
    \end{column}

    \begin{column}{0.65\paperwidth}
        \begin{figure}
            \begin{center}
                \includegraphics<1-2>[height=0.6\textwidth]{Screenshot2.png}
                \includegraphics<3-5>[height=0.6\textwidth]{Screenshot2_crash.png}
                \includegraphics<6->[height=0.6\textwidth]{crash.png}
                \caption{Visit Setup}
            \end{center}
        \end{figure}
    \end{column}
\end{columns}
\end{frame}

%4
\begin{frame}
\frametitle{So what's it look like?}
\begin{columns}
% Left
\begin{column}{0.29\paperwidth}
\begin{itemize}
    \item Lots of low frequency data underneath the surface
    \item Lots of high frequency data on the surface
    \item Can we try to reduce the data size and focus on the important stuff?
\end{itemize}
\end{column}
% Right
\begin{column}{0.70\paperwidth}
\begin{figure}[h!]
\begin{center}
    \movie[autostart,showcontrols,loop]{\includegraphics[height=0.70\textheight]{earthquake.png}}{earthquake.mp4}
    \caption{Raw data plotted}
\end{center}
\end{figure}
%\vspace{-2em}
%\Split
%{
%    \vspace{-3em}
%    \begin{figure}[h!]
%    \begin{center}
%        %\movie[autostart,showcontrols,loop]{\includegraphics[height=0.75\textheight]{earthquake.png}}{earthquake.mp4}
%        \movie[autostart,showcontrols,loop]{\includegraphics[height=0.35\textheight]{earthquake.png}}{earthquake.mp4}
%        %\movie[autostart,showcontrols,loop]{\includegraphics[height=0.35\textheight]{earthquake_rotate.png}}{earthquake_rotate2.mp4}
%        \caption{Raw data plotted}
%    \end{center}
%    \end{figure}
%}
%{
%    \vspace{-2em}
%    \begin{figure}[h!]
%    \begin{center}
%        %\movie[autostart,showcontrols,loop]{\includegraphics[height=0.75\textheight]{earthquake.png}}{earthquake.mp4}
%        \movie[autostart,showcontrols,loop]{\includegraphics[height=0.35\textheight]{earthquake_rotate.png}}{earthquake_rotate2.mp4}
%        \caption{Raw data plotted}
%    \end{center}
%    \end{figure}
%}
\end{column} % Right
\end{columns}
\end{frame}

%   \begin{frame}
%   \frametitle{Let's investigate}
%   \begin{columns}
%   \begin{column}{0.49\paperwidth}
%   \begin{figure}[h!]
%   \begin{center}
%   \includegraphics[height=0.40\textheight]{Screenshot.png}
%   \end{center}
%   \end{figure}
%   \end{column}

%   \begin{column}{0.49\paperwidth}
%   \begin{figure}[h!]
%   \begin{center}
%       %\movie[autostart,showcontrols,loop]{\includegraphics[height=0.75\textheight]{earthquake.png}}{earthquake.mp4}
%       \movie[autostart,showcontrols,loop]{\includegraphics[height=0.59\textheight]{earthquake_rotate.png}}{earthquake_rotate2.mp4}
%       \caption{Rotating figure at timestep 15}
%   \end{center}
%   \end{figure}
%   \end{column}
%   \end{columns}
%   \end{frame}

%5 
\begin{frame}
\frametitle{0.01 Threshold}
\begin{figure}[h!]
    \movie[autostart,showcontrols,loop]{\includegraphics[height=0.75\textheight]{earthquake_0p01thresh.png}}{earthquake_0p01thresh.mp4}
    \caption{0.01 Threshold Applied}
\end{figure}
\end{frame}
%6
\begin{frame}
\frametitle{0.02 Threshold}
\begin{figure}[h!]
    \movie[autostart,showcontrols,loop]{\includegraphics[height=0.75\textheight]{earthquake_0p02thresh.png}}{earthquake_0p02thresh.mp4}
    \caption{0.02 Threshold Applied}
\end{figure}
\end{frame}

%7
\begin{frame}
\frametitle{What about Ray Tracing the image?}
\vspace{-1.4em}
\begin{figure}[h!]
    \movie[autostart,showcontrols,loop]{\includegraphics[height=0.79\textheight]{earthquake.png}}{earthquake_max2_highres_0p02thresh.mp4}
    \caption{0.02 Threshold Ray Tracing}
\end{figure}
\end{frame}
%8
\begin{frame}
\frametitle{More rays equals more better?}
\vspace{-1.4em}
\begin{figure}[h!]
    \movie[autostart,showcontrols,loop]{\includegraphics[height=0.79\textheight]{earthquake.png}}{high_res.mp4}
    \caption{Failure because of timeout (took over 8hrs to get just this)}
\end{figure}
\end{frame}

%
\begin{frame}
\frametitle{More rays equals more better?}
\vspace{-1.4em}
\begin{figure}[h!]
    \movie[autostart,showcontrols,loop]{\includegraphics[height=0.79\textheight]{earthquake.png}}{highres_raycasting.mp4}
    \caption{0.02 Threshold High Resolution Ray Tracing}
\end{figure}
\end{frame}

\begin{frame}
\frametitle{Where to next}
    \begin{itemize}
        \item We need to get this running in Ascent now.
        \item Provide an in situ method for performing the same features.
        \item Requires restructuring the data so that we can use VTK-h filters.
        \item We need to extend features in both VTK-m and VTK-h.
        \item Make pretty pictures.
        \item Deliver product to scientists and make them happy.
    \end{itemize}
\end{frame}
% End
\end{document}
