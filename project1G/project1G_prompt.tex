\documentclass[12pt,letter]{article}
\usepackage{geometry}\geometry{top=0.75in}
\usepackage{amsmath}
\usepackage{amssymb}
\usepackage{mathtools}
\usepackage{xcolor}	% Color words
\usepackage{cancel}	% Crossing parts of equations out
\usepackage{tikz}    	% Drawing 
\usepackage{listings}   % For making code snippets
\usepackage{hyperref}   % Links

% Don't indent
\setlength{\parindent}{0pt}
% Function to replace \section with a problem name specifically formatted
\newcommand{\problem}[1]{\vspace{3mm}\Large\textbf{{Problem {#1}\vspace{3mm}}}\normalsize\\}
% Formatting function, like \problem
\newcommand{\ppart}[1]{\vspace{2mm}\large\textbf{\\Part {#1})\vspace{2mm}}\normalsize\\}
% Formatting 
\newcommand{\condition}[1]{\vspace{1mm}\textbf{{#1}:}\normalsize\\}

\lstset{frame=tb,
        language=bash,
        aboveskip=3mm,
        belowskip=3mm,
        showstringspaces=true,
        columns=flexible,
        basicstyle={\small\ttfamily},
        numbers=none,
        keywordstyle=\color{blue},
        stringstyle=\color{green},
        breaklines=true,
        breakatwhitespace=true,
        tabsize=3
}
\hypersetup{colorlinks=true,
            linkcolor=blue,
            filecolor=magenta,
            urlcolor=cyan
}

\begin{document}
\title{CIS 441/551: Project \#1G\\
       \large Gray Scale Shading with CUDA and Sobel Filtering}
\date{}
\maketitle

\large{\textbf{Instructions:}}
You will be converting an image to grey scale using CUDA 10.
\begin{enumerate}
    \item You can use the skeleton code provided to you, called project1G.cu
    \item You will be responsible for handling all memory allocations to and 
          from the device. Don't forget to free memory!
    \item You will also be responsible for writing the grey scale shader. You
          must use the correct luminosity equation (see below)
    \item Comments are provided to you in the code to help you out. Read over
          them before you begin typing. You can change any code you wish, as
          long as your image is the same, \textbf{BUT} you only need to add
          code and the comments will tell you where to place certain things,
          ensuring that you have the correct order of operations (this is
          very important).
    \item After you have successfully created the greyscale image you will
          want to start working on the Sobel Operator. The x and y kernels
          that you should use are already provided to you. You will implement
          the correct math operation using the correct indexes. You will have 
          to do a double for loop to get this working. The rest is provided for
          you. Figuring out the right indexing is key here.
    \item You should not expect a pixel perfect image. Expect the reference
          images to match by eye. 
\end{enumerate}

You will turn in a tarball with your code. Your code should generate two images:
a greyscale image and a Sobel image.

\large{\textbf{Hints:}}
\textbf{Where to start?} There is a lot of documentation around for cuda. I 
would suggest searching the internet and find a source that makes sense to you
(sometimes it just takes another source). Some helpful links are 
\href{https://www.nvidia.com/docs/IO/116711/sc11-cuda-c-basics.pdf}{Nvidia's
documentation} and 
\href{http://supercomputingblog.com/cuda/cuda-tutorial-1-getting-started/}
{The Supercomputing Blog}. Both these resources contain all the information
needed to complete the task. 


\textbf{Luminosity}: The human eye does not see all colors with the same 
magnitude. The center of our color vision is near yellow-green, and colors
closer to that will appear brighter. Thus we can't just average the RGB colors
and get an accurate looking image (try it). The correct formula to use is
\begin{equation}
    L = 0.21 * Red + 0.72 * Green + 0.07 * Blue
\end{equation}
You can see here that we have higher preferences to green and the lowest 
preference to blue, just like in the natural world. 


\textbf{Sobel Operator}: The \href{https://en.wikipedia.org/wiki/Sobel_operator}
{Sobel Operator} is frequently used in image detection because it highlights
edges. Usually you will want to greyscale your image first.


\textbf{Compiling}: Compile with the command
\begin{lstlisting}
nvcc -o project1G{,.cu} `pkg-config --cflags --libs opencv`
\end{lstlisting}
This command will work on Ubuntu systems. If you are using another system then
you will need to add the same cflags and libs that opencv requires. 
Note that we are using the back-tick, `, which is (probably) located at the top
left of your keyboard, left of the 1 key.


\textbf{Getting a GPU}: If you don't have access to a GPU please email
swalton2@uoregon.edu with the title "Alaska Access: 441" and an account will
be created for you and instructions will be provided on how to access the GPU.


\textbf{Checking with CPU}: If you decide to check your result with a CPU you
will not get a pixel perfect representation. It is normally a good idea to check
results like this, but your images will have differences unless you take into
account FMA instructions (on the GPU). More documentation can be found in
the \href{https://docs.nvidia.com/cuda/floating-point/index.html#axzz42SnDmIrm}{cuda documentation}. You can also email Steven if you need to edit your code to 
perform this check. 
\end{document}
